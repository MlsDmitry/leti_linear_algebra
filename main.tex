\documentclass[12pt,a4paper]{report}

\usepackage[T2A,T1]{fontenc}
\usepackage[utf8]{inputenc}
\usepackage[english, russian]{babel}
\usepackage{amsmath}
\usepackage{titlesec}
\usepackage{import}
\usepackage{xifthen}
\usepackage{pdfpages}
\usepackage{transparent}
\usepackage{physics}
\usepackage{amsthm}

\newtheorem{definition}{Определение}

\newcommand{\incfig}[1]{%
    \def\svgwidth{\columnwidth}
    \import{./figures/}{#1.pdf_tex}
}

\begin{document}
\begin{titlepage}
    \begin{center}
        \textbf{Курс аналитической геометрии и линейной алгебры}
    \end{center}
\end{titlepage}


\chapter{Матрица}

\begin{definition}
    Матрица размера $m \times n$ называется совокупность n * m элементов некоторого множества, записанных в виде прямоугольной таблицы, содержащей m строк и n столбцов.
\end{definition}

\begin{itemize}

    \item Если элементами матрицы являются числа, то матрица называется числовой.
        $$
        \begin{bmatrix}
            a_{11} & a_{12} & a_{1n} \\
            a_{21} & a_{22} & a_{2n} \\
            a_{31} & a_{32} & a_{3n} \\
        \end{bmatrix}
        $$


    \item Матрицу обозначают большими латинскими буквами A, B, C. Элементы матрицы малыми буквами, снабженные двумя индексами. Первый индекс - номер строки, второй - номер столбца, на пересечении которых находится элемент матрицы.

    \item Множество всех матриц, размера $m \times n$, обозначается \it{M}.

    \item Матрица размера $1 \times n$ называется матрицей строкой.
        $
        \begin{bmatrix}
            a_1 & a_2 & a_3
        \end{bmatrix}
        $

    \item Матрица размера $m \times 1$ называется матрицей столбцом.
        $
        \begin{bmatrix}
            a_1 \\
            a_2 \\
            a_3
        \end{bmatrix}
        $

    \item Матрица в которой m = n называется квадратной матрицей, порядка n.

    \item Матрица, все элементы которой равны нулю, называется нулевой и обозначается $\theta$.

    \item Матрица, на главной диагонали которой стоят какие-то числа, а все остальные элементы равны нулю, называется диагональной.
        $
        \begin{bmatrix}
            x_1 & 0 & 0 \\
            0 & x_2 & 0 \\
            0 & 0 & x_3
        \end{bmatrix}
        % \caption{диагональная матрица}
        $

    \item Матрица, на главной диагонали которой стоят единицы, обозначается буквой I(E) и называется единичной матрицей.

        $
        \begin{bmatrix}
            1 & 0 & 0 \\
            0 & 1 & 0 \\
            0 & 0 & 1
        \end{bmatrix}
        $

\end{itemize}

\section{Арифметические операции над матрицами}
Операции сложения и умножения на число называются линейными операциями.
\section{Сложение матриц}
\begin{gather*}
    A \in M_{m \times n}, B \in M_{m \times n} \\
    A + B = C,  C \in M_{m \times n}
\end{gather*}

\section{Умножение матрицы на число}

\begin{gather*}
    A \in M_{m \times n}, \lambda - number \\
    \lambda \cdot A = B \\
    b_{ij} = a_{ij} \cdot \lambda
\end{gather*}

\section{Свойства линейных операций}

\begin{enumerate}
    \item Свойство коммутативности $A + B = B + A$

    \item Свойство ассоциативности $(A + B) + C = A + (B + C)$

    \item $(\alpha + \beta) \cdot A = \alpha \cdot A + \beta \cdot B$

    \item $\alpha \cdot (A + B) = \alpha \cdot A + \alpha \cdot B$

\end{enumerate}
\section{Умножение матриц}

Матрицы можно умножать, если их размеры согласованы(количество столбцов первой матрицы равно количеству строк второй матрицы). В противном случае операция умножения не определена.

\begin{itemize}
    \item Пусть

        \begin{gather*}
            A \in M_{m \times n}, B \in M_{n \times k} \\
            A \cdot B = C, C \in M_{m \times k} \\
            c_{ij} = a_{ij} * b_{ij} + a_{i2} * b_{i2} + \\
        \end{gather*}

        \begin{figure}[ht]
            \centering
            \incfig{matrix-multiplication}
            \caption{matrix multiplication}
            \label{fig:matrix-multiplication}
        \end{figure}
    \item \begin{falign*}
            A =
            \begin{bmatrix}
                1 & 2 & 3 \\
            \end{bmatrix},
            B =
            \begin{bmatrix}
                4 \\
                0 \\
                2
            \end{bmatrix} \\
            A \cdot B = [ 10 ] \\
            B \cdot A = \begin{bmatrix}
                4 \\
                0 \\
                2
            \end{bmatrix}
            \cdot
            \begin{bmatrix}
                1 & 2 & 3
            \end{bmatrix}
            =
            \begin{bmatrix}
                4 & 8 & 12 \\
                0 & 0 & 0 \\
                2 & 4 & 6
            \end{bmatrix}
        \end{falign*}
    \item \begin{falign*}
            A =
            \begin{bmatrix}
                2 & 3 & -1 \\
                4 & 0 & 1
            \end{bmatrix},
            B =
            \begin{bmatrix}
                3 \\
                2 \\
                1
            \end{bmatrix}


            A \times B =
            \begin{bmatrix}
                11 \\
                13
            \end{bmatrix} \\

            B \times A \neq A \times B \\
        \end{falign*}
    \item \begin{falign*}
            A =
            \begin{bmatrix}
                0 & 0 \\
                1 & 1
            \end{bmatrix},
            B =
            \begin{bmatrix}
                1 & 1 \\
                1 & -1
            \end{bmatrix} \\
            A \cdot B =
            \begin{bmatrix}
                0 & 0 \\
                0 & 0
            \end{bmatrix} \\
            B \cdot A =
            \begin{bmatrix}
                1 & 1 \\
                -1 & -1
            \end{bmatrix}
        \end{falign*}
\end{itemize}
\end{document}
